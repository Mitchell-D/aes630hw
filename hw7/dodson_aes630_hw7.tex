% Questions on PDF page 134
\documentclass[12pt]{article}

\usepackage[utf8]{inputenc}
\usepackage[a4paper, margin=1in]{geometry}
\usepackage{booktabs}
\usepackage{physics}
\usepackage{amsmath}
\usepackage{amsfonts}
\usepackage{graphicx}
\usepackage{siunitx}
\usepackage{textcomp}

\graphicspath{{./figures}}

\title{Physical Climatology (AES 630) Homework 6 and 7}
\author{Mitchell Dodson}
\date{November 3, 2023}

\newcommand*{\problem}[2]{
    \begin{table}[ht]
    \centering
        \begin{tabular}{ | p{.1\linewidth} p{.9\linewidth} | }
            \hline
            \vspace{.3em}\textbf{\large#1:} & \vspace{.3em}\small{#2}\hspace{.2em}\vspace{.5em} \\ \hline
        \end{tabular}
    \end{table}
}

\begin{document}

\maketitle

\problem{10.3}{
    If the equilibrium is defined by $T_s = T_e \, \sqrt[4]{N+1}$ (from Section 3.8), then what is the sensitivity parameter $\lambda_R$ in Eq. 10.6 ($dT_s = \lambda_R dQ$)? First derive a formula, then compute a numerical value for $N=3$; Compare with Eq. 10.7 and explain the difference.
} % lambda = (N+1)^{1/4} / (4 sigma T_e^3)

\begin{equation}\label{q1}
    \begin{split}
        \frac{dT_s}{dQ} &= \lambda_R \\
        Q_{abs} &= \frac{S_0}{4} (1-\alpha_p) \\
        &= \sigma T_e^4 \\
        T_s &= T_e\sqrt[4]{N+1} \\
        \frac{dT_s}{dQ} &= \frac{1}{4} \left(\frac{Q(N+1)}{\sigma}\right)^{-\frac{3}{4}} \left(\frac{N+1}{\sigma}\right) \\
        &= \frac{1}{4} Q^{-\frac{3}{4}} \sqrt[4]{\frac{N+1}{\sigma}} \\
        &= \frac{1}{4} (\sigma T_e^4)^{-\frac{3}{4}} \sqrt[4]{\frac{N+1}{\sigma}} \\
        &= \frac{1}{4} \sigma^{-\frac{3}{4}} T_e^{-3} \sqrt[4]{\frac{N+1}{\sigma}} \\
        &= \frac{1}{4} \frac{\sqrt[4]{N+1}}{\sigma T_e^3} = \lambda_R \\
        \lambda_R(N=3) &= \frac{\sqrt{2}}{2} \frac{1}{\sigma T_e^3} \\
    \end{split}
\end{equation}

Equation \ref{q1} shows the derivation for the climate sensitivity parameter $\lambda_R$ given the $N$-layer model with no atmospheric absorption of shortwave radiation, and total longwave absorption at each layer.

\begin{equation}\label{eq10.7}
    (\lambda_R)_{BB} = \left(\frac{\partial(\sigma T_e^4)}{\partial T_s}\right)^{-1} = (4\sigma T_e^3)^{-1} = 0.26\,\si{K.(W.m^{-2})^{-1}}
\end{equation}

\problem{10.4}{
    Assume that transient eddies are the primary heat transport mechanism determining vertical and horizontal potential temperature gradients in mid-latitudes, and that the eddy transports respond to the mean gradients as indicated by Eqs. 10.31 and 10.33.
    If the radiative forcing of the meridional gradient produces a larger equator-to-pole heating contrast, how will the vertical gradient of potential temperature (static stability) in midlatitudes respond?
    Will the change in static stability be relatively large or small compared to the change in meridional temperature gradient?
    Assume that potential temperature is relaxed linearly toward a radiative equilibrium value and that the relaxation rate of vertical and horizontal gradients is the same.
} % The static stability will increase more than the meridional gradient.

\problem{10.5}{
    Use the Budyko energy-balance climate model (Eq. 10.23) to solve for the TSI as a function of iceline latitude. Use (10.14) and the parameters $B=1.45\,\si{W.m^{-2}.K^{-1}}$, $a_0 = 0.68$, $b_0 = 0.38$, and $1_2 = 0.0$. Plot the curves for the values $\gamma=3.8\,\si{W.m^{-2}.K^{-1}}$ and $\gamma = 2.8\,\si{W.m^{-2}.K^{-1}}$.
} %

\end{document}

\begin{figure}[h!]\label{q1q2}
    \centering
    \begin{tabular}{ c c c | c}
    \end{tabular}
\end{figure}

