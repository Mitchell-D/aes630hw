% Questions on PDF page 134
\documentclass[12pt]{article}

\usepackage[utf8]{inputenc}
\usepackage[a4paper, margin=1in]{geometry}
\usepackage{booktabs}
\usepackage{physics}
\usepackage{amsmath}
\usepackage{amsfonts}
\usepackage{graphicx}
\usepackage{siunitx}

\graphicspath{{./figures}}

\title{Physical Climatology (AES 630) Homework 4}
\author{Mitchell Dodson}
\date{October 8, 2023}

\newcommand*{\problem}[2]{
    \begin{table}[ht]
    \centering
        \begin{tabular}{ | p{.1\linewidth} p{.9\linewidth} | }
            \hline
            \vspace{.3em}\textbf{\large#1:} & \vspace{.3em}\small{#2}\hspace{.2em}\vspace{.5em} \\ \hline
        \end{tabular}
    \end{table}
}

\begin{document}

\maketitle


\problem{4.1}{
    If the top $100\,\si{m}$ of ocean warms by $5^\circ\si{C}$ during a 3-month summer, what is the average rate of net energy going into the ocean in $\si{W.m^{-2}}$?
    If the atmosphere warms by $20\,^\circ\si{C}$ over the same time, what is the rate of energy flux into the atmosphere?
    }% ANS: 270 Wm-2 , 26 Wm-2

\begin{equation}\label{q1_atmo_heat_cap}
    \begin{split}
        \bar{C}_A &= c_p p_s g^{-1} = \frac{1,004\,\si{J.K^{-1}.kg^{-1}}\cdot 101,320\,\si{Pa}}{9.81\,\si{m.s^{-2}}} \\
        &=\num{1.037e7}\,\si{J.K^{-1}.m^{-2}}
    \end{split}
\end{equation}
\begin{equation}\label{q1_ocean_heat_cap}
    \begin{split}
        \bar{C}_O &= \rho_w c_w d_w = 10^3\,\si{kg.m^{-3}} \cdot 4,218\,\si{J.kg^{-1}.K^{-1}} \cdot d_w \\
        &= d_w \cdot \num{4.218e6}\,\si{J.K^{-1}.m^{-3}}
    \end{split}
\end{equation}

Equations \ref{q1_atmo_heat_cap} and \ref{q1_ocean_heat_cap} model the average specific heat capacity of the entire atmosphere, and the volumetric thermal capacity of the first $d_w$ meters in depth of the ocean.

The summer months (June, July, and August) have (30, 31, 31) days respectively, so the time period consists of $\Delta t = (30+31+31)\cdot 24 \cdot 60 \cdot 60 = 7,948,800\,\si{s}$.

\begin{equation}\label{q1_atmo_rate}
    F_A = \frac{\bar{C}_A\,\Delta T}{\Delta t} = \frac{\num{1.037e7}\,\si{J.K^{-1}.m^{-2}}\,20\cdot\si{K}}{\num{7.9488e6}\,\si{s}} = 26\,\si{W.m^{-2}}
\end{equation}

\begin{equation}\label{q1_ocean_rate}
    F_O = \frac{\bar{C}_O\,\Delta T\,d_w}{\Delta t} = \frac{\num{4.218e6}\,\si{J.K^{-1}.m^{-3}}\cdot 5\,\si{K}\cdot 100\,\si{m}}{\num{7.9488e6}\,\si{s}} = 265.32\,\si{W.m^{-2}}
\end{equation}

The mean heating contributed to the atmosphere during the 3-month season is modeled by Equation \ref{q1_atmo_rate}, and Equation \ref{q1_ocean_rate} shows the mean heating of the ocean surface down to depth $d_w= 100\,\si{m}$. Although the above value $F_O = 265.32\,\si{W.m^{-2}}$ doesn't match the texbook's reported value of $F_O = 270\,\si{W.m^{-2}}$, assuming that all three months have 30 days returns the expected value of $~270\,\si{W.m^{-2}}$

\clearpage

\problem{4.4}{
    The surface blackbody emission can be linearized with respect to reference temperature $T_0$, $\sigma T_s^4 \approx \sigma T_0^4 + 4\sigma T_0^3(T_s - T_))+\cdots$.
    Similarly, the sensible surface cooling can be approximated $SH \approx c_p \rho C_D | u | (T_s - T_a)+\cdots$.
    Calculate and compare longwave emission and sensible heat flux rates of change wrt constant $T_s$ given:
    $T_0 = 288\,\si{K}$, $\rho = 1.2\,\si{kg.m^{-3}}$, $c_p = 1004\,\si{J.kg^{-1}K^{-1}}$,  $C_D = \num{2e-3}$, $|u| = 5\,\si{m.s^{-1}}$.
    }% ANS: LW: 5.4 Wm-2K-1 , SH: 12 Wm-2K-1

\begin{equation}\label{q4_semis}
    \begin{split}
        \sigma T_s^4 &\approx F^{\uparrow s} := \sigma T_0^4 + 4\sigma T_0^3 (T_s - T_0) + \cdots \\
        \frac{dF^{\uparrow s}}{d T_s} &= 4\sigma T_0^3 = 4\sigma\cdot(288\,\si{K})^3 = 5.4182\,\si{W.m^{-2}.K^{-1}} \\
    \end{split}
\end{equation}

\begin{equation}\label{q4_sshf}
    \begin{split}
        SH &\approx c_p \, \rho \, C_D \, |u| \, (T_s - T_a) + \cdots \\
        \frac{d SH}{d T_s} &= c_p \, \rho \, C_D \, |u|  = (1005\,\si{J.kg^{-1}.K^{-1}} \cdot 1.2 \, \si{kg.m^{-3}} \cdot \num{2e-3} \cdot 5\,\si{m.s^{-1}}) \\
        &= 12.048 W\,\si{m^{-2}.K^{-1}} \\
    \end{split}
\end{equation}

Equations \ref{q4_semis} and \ref{q4_sshf} are linearized approximations of the longwave surface emissivity and the sensible heat flux from the surface, respectively.
Since both surface heat sinks are expressed in terms of only one variable, their derivative with respect to that variable is a scalar value that can be solved for directly,
as shown in the equations. These results indicate that near the reference temperature $T_0 = 288\,\si{K}$, per Kelvin increase in surface temperature, the amount of heat lost to sensitive heat flux increases more than twice as fast as the amount of heat lost to longwave emission.

\problem{4.5}{
    Air with temperature $T_a = 27\,^\circ\si{C}$ moves across a dry parking lot with $|u|=5\,\si{m.s^{-1}}$.
    Surface solar insolation is $S^\downarrow = 600\,\si{W.m^{-2}}$, and the surface downward longwave radiation is $F^\downarrow(0)=300\,\si{W.m^{-2}}$.
    The longwave emissivity of the surface is $\epsilon_L = 0.85$, and the shortwave albedo is $\alpha_S = 0.1$.
    \textbf{(1)} What is the surface temperature at radiative equilibrium?
    \textbf{(2)} What is the surface temperature if the asphalt is instead concrete with $\alpha_S = 0.3$ (same $\epsilon_L$)?
    The air density and drag coefficient are as in problem 3.
    \textit{Hint:} Linearize the blackbody emission around the air temperature and use sfc energy to show (eq in text).
    }% ANS: asphalt: 50.5 C , concrete: 43.5 C

\begin{equation}\label{q5_linearize}
    \epsilon\sigma T_s^4 \approx \epsilon\sigma T_a^4 + 4\sigma\epsilon T^3_a (T_s - T_a )
\end{equation}

\begin{equation}\label{q5_ebal}
    \begin{split}
        -\epsilon\sigma T_s^4 &= S_s^\downarrow (1-\alpha_s) + (1-\epsilon) F^\downarrow_s -  c_p\rho c_D u (T_s - T_a) \\
        -\epsilon\sigma T_s^4 - 4\sigma\epsilon T^3_a (T_s - T_a )&=  S_s^\downarrow (1-\alpha_s) + (1-\epsilon) F^\downarrow_s -  c_p\rho c_D u (T_s - T_a) \\
        (T_s - T_a )[c_p\rho C_D u - 4\sigma T^3_a \epsilon] &=  S_s^\downarrow (1-\alpha_s) + \epsilon (F^\downarrow_s - \sigma T_a^4) \\
        T_s &=  \frac{S_s^\downarrow (1-\alpha_s) + \epsilon (F^\downarrow_s - \sigma T_a^4)}{[c_p\rho C_D u - 4\sigma T^3_a \epsilon]} + T_a
    \end{split}
\end{equation}

Equation \ref{q5_linearize} shows the linearization approximation of surface emissivity in terms of surface temperature, and Equation \ref{q5_ebal}
uses this approximation to solve the surface energy balance equation in terms of surface temperature $T_s$.

\begin{equation}\label{q5_sol1}
    \begin{split}
        T_s &= \frac{600\,\si{W.m^{-2}}(1-.1)+0.85\,(300\,\si{K}-\sigma(300\,\si{K})^4)}{1005\,\si{J.kg^{-1}.K^{-1}}\cdot1.2\,\si{kg.m^{-3}}\cdot \num{2e-3} \cdot 5\,\si{m.s^{-1}} + 4\cdot .85\sigma\cdot(300\,\si{K})} + 300\,\si{K} \\
        T_s &= 323.43\,\si{K}\text{ or } 50.434\,^\circ\si{C} \\
    \end{split}
\end{equation}

\begin{equation}\label{q5_sol2}
    \begin{split}
        T_s &= \frac{600\,\si{W.m^{-2}}(1-.3)+0.85\,(300\,\si{K}-\sigma(300\,\si{K})^4)}{1005\,\si{J.kg^{-1}.K^{-1}}\cdot1.2\,\si{kg.m^{-3}}\cdot \num{2e-3} \cdot 5\,\si{m.s^{-1}} + 4\cdot .85\sigma\cdot(300\,\si{K})} + 300\,\si{K} \\
        T_s &= 316.48\,\si{K}\text{ or } 43.48 \,^\circ\si{C} \\
    \end{split}
\end{equation}

Solving for the only unknown, surface temperature, at both $\alpha_s = 0.1$ and $\alpha_s = 0.3$, we recieve the results reported in Equations \ref{q5_sol1} and \ref{q5_sol2}.
The surface temperature is lower when the albedo is higher, which makes sense because the surface layer is absorbing a smaller fraction of incident insolation.


\problem{4.6}{
    \textbf{(1)} Do problem 4, but with a parking lot with the surrounding air at saturation, and including the effect of latent heat on the surface.
    Ignore the effects of surface water on the albedo.
    \textbf{(2)}Compare the surface temperature for wet and dry surfaces.
    \textbf{(3)} How would the results differ if air wasn't saturated?
    \textit{Hint:} Use Equation (4.34)
    }% ANS: dry: 50.5 C , wet: 33 C with B_e = 0.25

\begin{equation}\label{mix_ratio}
    q^*(T) := \frac{.622 e_s}{P - e_s}
\end{equation}

\begin{equation}\label{bolton1980}
    e_s(T) \approx 6.112\exp\left(\frac{17.67\,(T-273.15)}{T-29.65}\right)
\end{equation}

\begin{equation}\label{bowen}
    B_e = \frac{c_p}{L} \left(\frac{\partial q^*}{\partial T} \right) |_{T=T_s}
\end{equation}

\begin{equation}\label{q6_sh}
    \text{SH} = c_p \rho C_{DS} u (T_s - T_a)
\end{equation}

\begin{equation}\label{dq_dT}
    \frac{\partial q^*}{\partial T} \approx q^*(T) \left(\frac{L}{R_vT^2}\right)
\end{equation}

\begin{equation}\label{q6_le}
    \begin{split}
        \text{LE} &= L\rho c_{DL} u (q_s - q_a) \\
        &= L\rho c_{DL} u (q_s^*(T_s)(1-\text{RH})+\frac{c_p\,\text{RH}}{L\,B_e}(T_s-T_a)) \\
        (\text{as RH }\to 1) \,\,\,\,\text{LE} &= L\rho C_{DL} u \frac{c_p(T_s-T_a)}{L\,B_e} \\
    \end{split}
\end{equation}

Equation \ref{mix_ratio} shows the saturation mixing ratio in terms of the saturation vapor pressure, which is
estimated empirically in Equation \ref{bolton1980}, which was adapted from (Bolton, 1980). Equation \ref{bowen} shows
the equation for the equilibrium Bowen ratio.

Equations \ref{q6_sh} and \ref{q6_le} show the energy equations for sensible and latent heat fluxes from the surface.
Since we're considering a saturated atmosphere, the surface energy balance is:
$0 = S_0^\downarrow (1-\alpha_s) + (1-\epsilon_s) F_s^\downarrow + \epsilon F_s^\uparrow - SH - LH$

\begin{equation}\label{latent_ebal}
    T_s = \frac{S_s^\downarrow (1-\alpha_s) + \epsilon (F^\downarrow_s - \sigma T_a^4)}{(c_p + \frac{dq^*}{dT}|_{T=T_s}) \rho C_D u - 4\sigma T^3_a \epsilon} + T_a
\end{equation}

Substituting this to the energy balance in Equation \ref{q5_ebal} to include the additional term for sensible heat, and making the assumption that the sensible and latent drag coefficients are equal yields Equation \ref{latent_ebal}

\begin{equation}\label{q6_dry}
    \begin{split}
        T_s &= \frac{600(1-.1)+.85(300-300^4\sigma)}{(1005+0)\cdot1.2\cdot.002\cdot 5 + 4\sigma\,.85\cdot300^3} + 300\,\si{K} \\
        %&= \frac{600(1-.1)+.85(300-300^4\sigma)}{(1005+\frac{dq^*}{dT}|_{T=T_s})\cdot1.2\cdot.002\cdot 5 + 4\sigma\,.85\cdot300^3} + 300\,\si{K} \\
        &= 323.4\,\si{K} = 50.434\,^\circ\si{C}
    \end{split}
\end{equation}

Using Equation \ref{latent_ebal} and the mixing ratio approximation \ref{dq_dT} for $T_s = 300K$ and $P_s = 1013.2\,\si{mb}$, and assuming a dry surface (zero mixing ratio), Equation \ref{q6_dry} shows that the equilibrium dry-surface temperature is about $50.434\,^\circ\si{C}$, which is the same as the answer received in the previous question since there is no increase in the rate of of heat loss from the latent energy of evaporation.

\begin{equation}\label{q6_mr}
    \frac{\partial q^*}{\partial T} \approx q^*(T) \left(\frac{L}{R_vT^2}\right) = \frac{e_s\,.622\cdot\num{2.5e6}}{(1013.2-e_s)(461.5\cdot300^2)} = 1.3656\,\si{K^{-1}}
\end{equation}

The rate of change in saturation mixing ratio is calculated with Equations \ref{mix_ratio}, \ref{bolton1980}, and \ref{dq_dT}, which enable us to use the energy balance represented by Equation \ref{latent_ebal} to calculate the surface temperature, as shown in Equation \ref{moist_stemp}.

\begin{equation}\label{moist_stemp}
    \begin{split}
        T_s &= \frac{600(1-.1)+.85(300-300^4\sigma)}{(1005+\num{5.914e3})\cdot1.2\cdot.002\cdot 5 + 4\sigma\,.85\cdot300^3} + 300\,\si{K} \\
        &= 306.95\,\si{K} = 33.948\,^\circ\si{C}
    \end{split}
\end{equation}

If the air was unsaturated, the amount of evaporative cooling would increase proportionally in order to re-establish equilibrium, increasing the Bowen ratio. This has the effect of increasing the amount of latent energy, thus further decreasing the surface temperature.

\end{document}
